\documentclass{article}

% Language setting
% Replace `english' with e.g. `spanish' to change the document language
\usepackage[english]{babel}

% Set page size and margins
% Replace `letterpaper' with `a4paper' for UK/EU standard size
\usepackage[letterpaper,top=2cm,bottom=2cm,left=3cm,right=3cm,marginparwidth=1.75cm]{geometry}

% Useful packages
\usepackage{amsmath}
\begin{document}

\title{Computing thruster forces given desired linear acceleration and angular acceleration using linear programming}
\maketitle
$M$ is mass and $I$ is the inertia tensor.

$a$ is linear acceleration and $\alpha$ is angular acceleration.

$a_T$ is the target linear acceleration and $\alpha_T$ is the target angular acceleration.

$\tau_i$ is the maximum torque and $f_i$ is the maximum force generated by the engine. 

$c_i$ is the Specific Impulse (Efficiency) of the engine.

$x_i$ is the activation of that engine. Has a range of [0,1].

$ a_T, \alpha_T, a_i, \alpha_i, \tau_i,$ and $f_i$ are $3$x$1$ vectors. $x_i$ and $c_i$ are scalars.

$ a, \alpha, \tau, $and $f$ are $3$x$N$ matrices. $x$ and $c$ are $N$x$1$ vectors. $N$ is the number of engines.
\\
\\
We know that $force = mass \text{ }\cdot\text{ } acceleration$ and $torque = inertia\text{ }tensor  \text{ }\cdot\text{ } angular\text{ } acceleration$
\\
\\
Rearranging the formulas and using our values we arrive at
$$ a = M^{-1} f x$$
$$ \alpha = I^{-1} \tau x $$
\\
We concatenate the matrices so that we can express the problem with linear programming
$$ A = \begin{bmatrix}
a & \alpha
\end{bmatrix} $$

$$ b = \begin{bmatrix}
a_T & \alpha_T
\end{bmatrix} $$
\\
As such, the linear program can be expressed as

Find a vector $\mathbf{x}$

that minimizes $ \mathbf{c^Tx} $

subject to $\mathbf{Ax = b}$

and $\mathbf{0\geq x \geq 1}$\\
\\
Thus by solving the linear program, we find our activation vector $x$ that tells each engine how much it should activate to achieve a target linear acceleration $a_T$ and a target angular acceleration $\alpha_T$.

\end{document}