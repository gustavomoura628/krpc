\documentclass{article}

% Language setting
% Replace `english' with e.g. `spanish' to change the document language
\usepackage[english]{babel}

% Set page size and margins
% Replace `letterpaper' with `a4paper' for UK/EU standard size
\usepackage[letterpaper,top=2cm,bottom=2cm,left=3cm,right=3cm,marginparwidth=1.75cm]{geometry}

% Useful packages
\usepackage{amsmath}
\begin{document}

\title{Computing thruster forces given desired linear acceleration and angular acceleration using inertia tensors}
\maketitle
$M$ is mass and $I$ is the inertia tensor.

$a$ is linear acceleration and $\alpha$ is angular acceleration.

$a_T$ is the target linear acceleration and $\alpha_T$ is the target angular acceleration.

$\tau_i$ is the maximum torque and $f_i$ is the maximum force generated by the engine. 

$x_i$ is the activation of that engine. Has a range of [0,1].

cdot ( $\cdot$ ) will be used to denote element wise multiplication.

$$ a = \sum_i M^{-1} x_i \cdot f_i$$
$$ \alpha = \sum_i I^{-1} x_i \cdot \tau_i $$

$$ objective(x, a_T, \alpha_T, I, M, \tau, f) = |a - a_T|  + |\alpha - \alpha_T| $$

The objective function will be called a lot, so we should minimize redundant calculations as much as possible.

$$ a^{max}_i = M^{-1} f_i$$
$$ \alpha^{max}_i = I^{-1} \tau_i $$

Thus the objective function can be rewritten as

$$ objective(x, a_T, \alpha_T, a^{max}, \alpha^{max}) = |x^T a^{max} - a_T|  + |x^T \alpha^{max} - \alpha_T| $$


\end{document}